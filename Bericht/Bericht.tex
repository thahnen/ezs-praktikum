\documentclass[10pt,a4paper]{scrartcl}

\usepackage[ngerman]{babel}			% N�tig f�r deutsche Silbentrennung
\usepackage[ansinew]{inputenc}		% N�tig f�r deutsche Umlaute
\usepackage{scrpage2}				% N�tig f�r Fusszeile
\usepackage{graphicx}				% N�tig f�r eingebettete Bilder
\usepackage{hyperref}
\usepackage{listings}				% N�tig f�r Quelltext-Einbindung
\usepackage{courier}				% N�tig f�r Courier-Font (Quelltext)
\usepackage[bottom=10em]{geometry}	% N�tig f�r Gr�ssenver�nderung Fusszeile

\clearscrheadfoot
\pagestyle{plain}

% Geforderte "Formatvorlage" f�r Autoren; Parameter:
% - 1) Vor- & Zuname
% - 2) Matrikelnummer
% - 3) Gruppennummer
\newcommand{\autor}[3]{#1, \\ & & Matrikelnummer #2, \\ & & Gruppe #3}

% Geforderte "Formatvorlage" f�r Quellcode; Parameter:
% - 1) Sprache des Quellcodes (siehe: https://en.wikibooks.org/wiki/LaTeX/Source_Code_Listings#Supported_languages)
% - 2) Zeilengr�sse (nicht Schriftgr�sse!), Standard: 10
%
% Muss gefolgt werden mit:
% \begin{lstlisting}[frame=single]
% ... Code ...
% end{lstlisting}
\newcommand{\code}[2]{\lstset{language=#1, basicstyle=\fontsize{8}{#2}\selectfont\ttfamily}}


\begin{document}


% ================================================================================
%
%		Titelblatt (Institution, Titel/ Untertitel, Studiengang/ Semester, Autor, Datum, Version)
%
% ================================================================================
\begin{center}
	\LARGE{Hochschule Niederrhein} \\
	\Large{Fachbereich Elektrotechnik und Informatik} \\
	\large{Labor f\"ur Echtzeitsysteme}
\end{center}
\begin{verbatim}







\end{verbatim}
\begin{center}
	\textbf{\huge{Praktikum Echtzeitsysteme}} \\
	\textbf{\LARGE{Termine 1-3}} \\
	\textbf{\Large{B-I-5}}
\end{center}
\begin{verbatim}







\end{verbatim}
\begin{center}
	\begin{tabular}{lll}
		\textbf{Autor:} & & \autor{Tobias Hahnen}{1218710}{C} \\
		& & \\
		\textbf{Datum:} & & \today \\
		& & \\
		\textbf{Version:} & & 1.0.0
	\end{tabular}
\end{center}


\newpage


% ================================================================================
%
%		Inhaltsverzeichnis (automatisch generiert)
%
% ================================================================================
\tableofcontents
\newpage


% ================================================================================
%
%		1. Termin
%
% ================================================================================
\section{Termin}

\newpage


% ================================================================================
%
%		2. Termin
%
% ================================================================================
\section{Termin}

\newpage


% ================================================================================
%
%		3. Termin
%
% ================================================================================
\section{Termin}

\newpage


% ================================================================================
%
%		3) Zusammenfassung Praktikum 1-3
%
% ================================================================================
\section{Zusammenfassung}

\newpage


% ================================================================================
%
%		4) Anhang
%
% ================================================================================
\section{Anhang}
\label{sec:anhang}

\newpage


\end{document}